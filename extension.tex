\documentclass{jsarticle}

\RequirePackage{amsmath,amssymb,amsthm, amscd, comment}
\input{../tex/theorems}
\input{../tex/symbols}
\title{加群の拡大}
\author{@unaoya}
\date{\today}                                           % Activate to display a given date or no date

\begin{document}
\maketitle
%\section{}
%\subsection{}

\section{加群の拡大}
$A$を可換環とする。
\begin{dfn}
$A$加群$M$とは、アーベル群$M$への$A$の作用が定まっているもの。
ここで$A$の作用とは、各$a\in A$について$a$倍写像という群準同型$a:M \to M$が定まっていて、
\begin{enumerate}
\item $(a+b)m=am+bm$
\item $(ab)m=a(bm)$
\item $1m=m$
\end{enumerate}
を満たすもの。
\end{dfn}

これは環の準同型$A \to \End(M)$を与えるという事である。
ここで$\End(M)$とは$M$のアーベル群の自己準同型$M \to M$全体の集合に、
和を$(f+g)(m)=f(m)+g(m)$で、積を射の合成$(fg)(m)=f(g(m))$で与えた環である。

例えば$A=k$が体であれば$k$加群は$k$ベクトル空間のことであり、$A$の作用とはスカラー倍の事である。

\subsection{$k[T]$加群}
$k$を体とし、$V$を$k$ベクトル空間とする。
この$V$に一変数多項式環$k[T]$加群の構造を定めよう。
ただし、ここでは$k[T]$の作用は、
\begin{enumerate}
\item $a\in k[T]$倍写像が$k$線形写像
\item $k\subset k[T]$の作用はもとのスカラー倍
\end{enumerate}
をみたすもののみを考えることにする。

これは各々の多項式$f(T)$に対して$k$線形写像$f(T):V \to V$を適切に定めるということ。
(群$G$の$k$線形表現は、各$g\in G$に対して$g:V \to V$という$k$線形写像が適切に定まっているもので、これと似たことを考えている。)

\begin{lem}
$k[T]$の$V$への作用は$T$の作用$T:V \to V$のみから決定される。
\end{lem}
つまり他の多項式の作用は、加群の定義から$T$の作用により自動的に決まるということ。
例えば
\begin{align*}
(T^2+3T)v=T^2v+3Tv=T(Tv)+3(Tv)
\end{align*}
などである。
結局、$k[T]$加群の構造は$T$の作用である$k$線形写像$T:V \to V$を定めることに他ならない。

$V$を$k$上の有限次元ベクトル空間とする。
このとき$V$の基底をとって考えると、$T$の作用$T:V \to V$は$k$線形写像なのである行列$A$で表示できる。
したがって$V$に$k[T]$加群の構造を定めることは、$T$に対応する行列$A$を定めることとと等価である。

\subsection{$k[T]$加群の拡大}
\begin{dfn}
$A$加群$M, N$に対し$N$の$M$による$A$加群の拡大とは、$A$加群$E$と以下のような$A$加群の短完全列
\[
\begin{CD}
0 @>>> M @>i>> E @>p>> N @>>> 0\\
\end{CD}
\]
のことをいう。
\end{dfn}

$V_1, V_2$を$1$次元$k$ベクトル空間とし、そこに$k[T]$加群の構造を定める。
それぞれ$T$作用を$a_1, a_2$倍で定めることにしよう。
$1$次元なので、これは実は基底の取り方によらずに決まっている。

\begin{prob}
$V_1$の$V_2$による$k[T]$加群の拡大
\[
\begin{CD}
0 @>>> V_2 @>i>> V @>p>> V_1 @>>> 0\\
\end{CD}
\]
が全部でどれだけあるか調べよう。
\end{prob}


このとき、これは$k$加群の完全列でもあるから、$V$は$k$上$2$次元の線形空間であり$k$加群としては$V=V_1\oplus V_2$と表せる。
この直和分解から定まる$V$の基底$e_1, e_2$を固定しよう。
つまり、$i:V_2 \to V$の像は$ke_2\subset V$であり、$p:ke_1 \to V_1$は同型になるものとする。

$k[T]$加群$V$の$T$倍作用がこの基底$e_2, e_1$について行列$A$で表示されるとしよう。
$V_2\subset V$は$k[T]$部分加群である、つまり$i:V_2 \to V$が$k[T]$加群の準同型であるから、
$i(Te_2)=Ti(e_2)=Te_2=a_2e_2$となる。
よって$i$は単射なので、$Te_2=a_2e_2$となり、
\begin{align*}
A=\begin{pmatrix}a&b\\0&a_2\end{pmatrix}
\end{align*}
と上三角行列になることがわかる。

さらに$Te_1=ae_1+be_2$とすると、
$p:V \to V_1$が$k[T]$準同型であるから$p(ae_1+be_2)=p(Te_1)=Tp(e_1)=Te_1=a_1e_1$より、$a=a_1$となることが条件。
よって
\begin{align*}
A=\begin{pmatrix}a_1&b\\0&a_2\end{pmatrix}
\end{align*}
としておけば、上の完全列が$k[T]$加群としての完全列である。

逆にこのような行列を用いて$V$に$k[T]$加群の構造を定めることができ、これが拡大を与えることもわかる。

つまり、$V$における$T$倍に対応する行列は上三角行列であり、対角成分が$V_1, V_2$から決定されることがわかる。

\subsection{拡大の同型類}
\begin{dfn}
二つの拡大の間の同型とは、$k[T]$加群の同型$\phi:V \to V'$で$V_1, V_2$には恒等写像を誘導するもの。
つまり、次の図式が可換になるもの。
\[
\begin{CD}
0 @>>> V_2 @>i>> V @>p>> V_1 @>>> 0\\
@.    @|    @V\phi VV     @|    \\
0 @>>> V_2 @>i'>> V' @>p'>> V_1 @>>> 0\\
\end{CD}
\]
\end{dfn}

ここでは$V'$は$k$加群としては$V$と一致していて、$k[T]$作用が異なるかもしれないと考える。

\begin{prob}
$V_1$の$V_2$による拡大$V, V'$に対して、上三角行列$A, A'$が対応することをみた。
この二つが拡大の同型となるための$A, A'$についての条件を求めよう。
\end{prob}
拡大の同型$\phi:V \to V'$は$k$加群の同型でもあり、これを$V, V'$の$k$上の基底$e_1, e_2$に対して行列$P$で表す。
上の条件は
\begin{enumerate}
\item $\phi(e_1)=e_1$
\item $\phi(e_2)=e_2 \mod V_2$
\end{enumerate}
という事なので、
\begin{align*}
P=\begin{pmatrix}1&q\\0&1\end{pmatrix}
\end{align*}
となる。
これが$k[T]$加群の同型なので$\phi(Tv)=T\phi(v)$であり、これを$V, V'$での$T$作用をそれぞれ
\begin{align*}
A=\begin{pmatrix}a_1&b\\0&a_2\end{pmatrix},~
A'=\begin{pmatrix}a_1&b'\\0&a_2\end{pmatrix}
\end{align*}
と表すとすると、$k[T]$加群の射であるという条件は$PA=A'P$となる。

したがって$P^{-1}AP=A'$であり、
\begin{align*}
\begin{pmatrix}1&-q\\0&1\end{pmatrix}\begin{pmatrix}a_1&b\\0&a_2\end{pmatrix}\begin{pmatrix}1&q\\0&1\end{pmatrix}
&=\begin{pmatrix}a_1&a_1q+b-a_2q\\0&a_2\end{pmatrix}\\
&=\begin{pmatrix}a_1&b'\\0&a_2\end{pmatrix}
\end{align*}
となる。

これを$a_1, a_2$が等しいか否かで場合分けして考える。
\begin{enumerate}
\item $a_1=a_2$であれば$a_1q+b-a_2q=b$なので、$q$をどのようにとっても$b'=b$となる。
つまり$b\neq b'$であれば拡大は同型とならない。
\item $a_1\neq a_2$のとき、$q=\dfrac{-b}{a_1-a_2}$とすれば$b'=0$となる。つまりこの場合には全ての拡大が同型ということになる。
\end{enumerate}

$V$の$k[T]$加群としての半単純化は$V_1\oplus (V/V_1)=V_1\oplus V_2$である。
つまり、上の拡大で与えられる$V$は全て半単純化すると$V_1\oplus V_2$と一致する。

一方、拡大の同型類は
\begin{enumerate}
\item $a_1=a_2$のとき、$k$と同じだけあり半単純化すると区別できない
\item $a_1\neq a_2$のとき、全ての拡大は同型であり半単純化しても変わらない
\end{enumerate}
となる。
\section{$\Ext^1$加群}
さて、上で見た拡大の分類をホモロジー代数を用いて求める方法を紹介しよう。

\begin{thm}
$A$を可換環とし、$M, N$を$A$加群とする。
$N$の$M$による$A$加群の拡大の同型類のなす集合$\Ext^1_A(N,M)$は次の方法で求めることができる。
\begin{enumerate}
\item $N$のprojective resolution
\begin{align*}
\cdots \to P_2 \to P_1 \to P_0 \to N \to 0
\end{align*}
をとる。
\item 上の$\Hom_A(-,M)$をとり、複体
\begin{align*}
0 \to \Hom_A(N,M) \to \Hom_A(P_0,M) \to \Hom_A(P_1,M) \to \Hom_A(P_2,M) \to \cdots
\end{align*}
を作る。
\item 上の複体のコホモロジーをとり、その
\begin{align*}
H^1(\Hom_A(P_\bullet,M))=\ker(\Hom_A(P_1,M) \to \Hom_A(P_2,M))/im(\Hom_A(P_0,M) \to \Hom_A(P_1,M))
\end{align*}
が$\Ext^1_A(N,M)$となる。
\end{enumerate}
\end{thm}
ここではこの定理の証明をする代わりに前節の例でこの計算をしてみる。
つまり、$A=k[T], M=V_2, N=V_1$として$\Ext^1_{k[T]}(V_1, V_2)$を計算してみよう。

\subsection{projective resolution}
\begin{dfn}
$N$のprojective resolutionとは、完全列$\cdots \to P_2 \to P_1 \to P_0 \to N \to 0$であって、
各$P_i$がprojective $A$-moduleであるものである。
\end{dfn}
projective moduleの定義は説明しないが、自由加群$A^n$はprojective $A$-moduleである。

\begin{prob}
$k[T]$加群$V_1$のprojective resolutionを作れ。
\end{prob}
まず$f\colon P_0=k[T] \to V_1$を$1 \mapsto e_1$とし$k[T]$準同型となるよう定める。
つまり$f(T)=a_1e_1$である。
このとき$\ker f=(T-a_1)$となる。

次に$g:P_1=k[T] \to P_0=k[T]$を$1\mapsto T-a_1$とし$k[T]$準同型となるよう定める。
するとこれは単射であり、像は$(T-a_1)\subset k[T]$と一致する。
よって
\[
\begin{CD}
0 @>>> P_1 @>g>> P_0 @>f>> V_1 @>>> 0\\
\end{CD}
\]
が$k[T]$加群の完全列となる。

$P_0=P_1=k[T]$はともにrank $1$の自由$k[T]$加群なので、これらはprojective moduleである。
つまり、これが$V_1$のprojective resolutionを与える。

\subsection{$\Hom_A(-,M)$}
$A$加群$M$が与えられたとき、$\Hom_A(-,M)$は$A$加群の圏から$A$加群の圏への関手を与える。
この関手は$A$加群$N$に対して$A$加群$\Hom_A(N,M)$を対応させ、$A$加群の射$f:N \to N'$に対して
$f^*:\Hom_A(N',M) \to \Hom(N,M); \phi \mapsto \phi\circ f$を対応させるもの。

上で得られたprejective resolutionに関手$\Hom_{k[T]}(-,V_2)$を作用させると
\[
\begin{CD}
0 @>>>  \Hom_{k[T]}(V_1,V_2) @>f^*>> \Hom_{k[T]}(P_0,V_2) @>g^*>> \Hom_{k[T]}(P_1,V_2) @>>> 0\\
\end{CD}
\]
という列が得られるが、これは$k[T]$加群の複体になる。

\begin{prob}
$g^*:\Hom_{k[T]}(P_0,V_2) \to \Hom_{k[T]}(P_1,V_2)$がどのような写像になるか具体的に計算してみよう。
\end{prob}
$\phi:P_0\to V_2\in \Hom_{k[T]}(P_0,V_2)$が与えられたとき、$g^*(\phi)=\phi\circ g\in\Hom_{k[T]}(P_1,V_2)$
は、
\begin{align*}
\phi\circ g(1)=\phi(T-a_1)=(T-a_1)\phi(1)=T\phi(1)-a_1\phi(1)=(a_2-a_1)\phi(1)
\end{align*}
となる。

ところで$\Hom_{k[T]}(k[T],V)$の元は$1$の行き先を決めれば決まってしまうので、
$\Hom_{k[T]}(k[T],V)$は$ev_1:\phi \mapsto \phi(1)$として$V$と同型になる。
この逆射を$v\mapsto \phi_v$と表す。
つまり$\phi_v\in \Hom_{k[T]}(k[T],V)$は$\phi_v(a)=av$となるもの。

これを使って上の写像$g^*:\Hom_{k[T]}(P_0,V_2) \to \Hom_{k[T]}(P_1,V_2)$を書き換える。
つまり次の図式
\[
\begin{CD}
\Hom_{k[T]}(P_1,V_2) @>g^*>>  \Hom_{k[T]}(P_1,V_2)\\
@Vev_1VV @Vev_1VV\\
V_2 @>g^*>> V_2
\end{CD}
\]
を可換にする$g^*:V_2 \to V_2$は$v\in V_2$に対して
\begin{align*}
g^*(v)=ev_1(g^*(\phi_v))=(\phi_v\circ g)(1)=\phi_v(g(1))=\phi_v(T-a_1)=(T-a_1)v=(a_2-a_1)v
\end{align*}
となる。

$V_2$は$k$上$1$次元であり、上の写像は$k$加群の準同型でもあるので、$a_2\neq a_1$の場合上の写像は全射かつ単射であり、$a_2=a_1$の場合には$0$射になる。

\subsection{$\Ext^1_A$}
上で得られた複体
\[
\begin{CD}
0 @>>> V_2 @>g^*>> V_2 @>>> 0\\
\end{CD}
\]
のcohomologyを計算しよう。
\begin{enumerate}
\item $a_1\neq a_2$の場合、射$g^*$は$0$射なので$H^1=\Ext^1(V_1, V_2)=k$となる。
\item $a_1=a_2$の場合、射$g^*$は同型なので$H^1=\Ext^1(V_1, V_2)=0$となる。
\end{enumerate}

これは前節の最後に計算した拡大の同型類の分類結果とうまく対応している。

\section{終わりに}
\begin{enumerate}
\item $\Ext^1$はprojective resolutionの取り方によらないか?
\item $\Ext^1$の各元が拡大にどのように対応しているか?
\item $\Ext^1$の群構造は何か?
\end{enumerate}
などの疑問が残るが、それについてはまた改めて。

実はこの方法はより一般的に、アーベル圏の間の右完全関手の左導来関手を計算する方法を与えている。
これの双対として左完全関手の右導来関手を計算でき、例えば群のコホモロジーを計算できる。
\end{document}