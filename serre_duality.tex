\documentclass{jsarticle}
\setlength{\textwidth}{\fullwidth}
\setlength{\evensidemargin}{\oddsidemargin}
\RequirePackage{amsmath,amssymb,amsthm, amscd, comment, multicol}
\usepackage[all]{xy}
\input{../tex/theorems}
\input{../tex/symbols}
\usepackage[dvipdfmx]{graphicx}
\usepackage{tikz}
\usepackage{tkz-euclide}
\usetkzobj{all}
\usetikzlibrary{intersections, calc}
\title{Serre duality}
\author{@unaoya}
\date{\today}

\begin{document}
\maketitle

参考文献はHartshorne

\section{Sheaf}

\section{Cohomology}
ここではコホモロジーの定義と、いくつかの性質を示す。

$F:C \to D$が適当な条件を満たす(左完全とか)アーベル圏の間の関手とする。
これに対して以下の条件を満たす関手$R^iF:C \to D$を定める。

$M \to I^\bullet$がinjective resolutionであれば$R^iFM = H^i(F(I^\bullet))$となる。

acyclic resolutionによる計算。
$A$がacyclicとは$R^iFA=0$となること。
$M \to A^\bullet$をacyclic resolutionとすると、$R^iFM = H^i(F(A^\bullet))$となる。


\section{Sheaf cohomology}
\begin{lem}
環付き空間$(X, O_X)$上の$O_X$-mod $F$がflasqueなら$\Gamma$-acyclicである。
\end{lem}
\begin{proof}
まず$O_X$-mod $I$がinjectiveならflasqueであることを示す。
$I(U) = Hom(O_X\vert_U, I\vert_U) = Hom(j_!(O_X\vert_U),I)$が$V \subset U$について関手的に成り立ち、
$(j_V)_!(O_X\vert_V) \to (j_U)_!(O_X\vert_U)$が単射であり$I$がinjectiveだからこれは全射。

$F$をinjective $I$に埋め込み、そのcokernelを$G$とする。
$F, I$がflasqueであることから$G$もflasqueである。

$0 \to F \to I \to G \to 0$の長完全列を考える。
$F$がflasqueなので$0 \to H^0(X, F) \to H^0(X,I) \to H^0(X,G) \to 0$が完全であり、$H^1(X, F)=0$となる。
$I$についてのコホモロジーが消えることに注意すると次数に関する帰納法から$H^i(X,F)=0$が言える。
\end{proof}


\section{Affine schemeのcohomology}
この節ではaffine schemeのcohomologyが消えることを証明する。
Noetherでない場合は?
\begin{thm}
$A$をNoether環、$M$を$A$-moduleとする。
$X =\Spec A$と$O_X$-mod $\cF=\tilde{M}$について、$i>0$に対し$H^i(X,\cF)=0$
\end{thm}

$0\to M \to I^\bullet$をinjective resolutionとする。
これに対し$0 \to \tilde{M} \to \tilde{I}^\bullet$がflasque resolutionになることを示す。

\begin{lem}
$I$がinjective $A$-modであるとき$\tilde{I}$はflasque $O_X$-mod
\end{lem}
\begin{proof}
まず
\begin{lem}
$A$がnoetherで$I$がinjectiveならば$I \to I_f$が全射である。
\end{lem}
を示す。
\begin{proof}
$x \in I_f$は$y \in I, n>0$を用いて$x=\dfrac{y}{f^n}$とかける。
$y=f^nz$とできればよい。

$(f^n) \subset A$について$(f^n) \to I$を$af^n \mapsto ay$と定めると、
$I$がinjectiveなので$A \to I$が定まり、$1$の像が$x$になる。

この写像が定義できるかはわからないが、$A$がNoetherより$f^r$のannihilator $b_r$を考えると
\end{proof}

$Y=\overline{Supp(\tilde{I})}$とする。
$Y$についてのNoetherian inductionにより証明する。

まず$Y$がclosed point $1$点からなる場合、$\tilde{I}$はskyscraper sheafなのでflasqueである。

open $U \subset X$に対して$\Gamma(X, \tilde{I}) \to \Gamma(U, \tilde{I})$が全射であることを証明する。
$D(f) \subset U$となるような$f$をとる。
このとき、上の補題から$\Gamma(X, \tilde{I})=I \to \Gamma(U,\tilde{I}) \to \Gamma(D(f), \tilde{I})=I_f$は全射。

$Z=X-D(f)$とし、$\Gamma_Z(X, \tilde{I}) \to \Gamma_Z(U, \tilde{I})$を考える。
$J=\Gamma_a(I)$にたいし$\Gamma_Z(U, \tilde{I})=\Gamma(U, \tilde{J})$である。
実際、$\Gamma_a(I)=\Gamma(X, \cH_Z^0(\cF))$であり、
$0 \to \cH^0_Z(\cF) \to \cF \to j_*(\cF\vert_U)$が完全だから$\cH^0_Z(\cF)$はq-cohなのでよい。

ここで$J$がinjectiveであれば、$Z$について帰納法の仮定からこの射が全射である。
$J$がinjectiveであることは次のように示せる。
イデアル$b \to A$からの射$\phi:b \to J$が$A \to J$に伸びればよい。
$A$がNoetherなのである$n>0$があって$a^n\phi(b)=0$である。
よってKrullの定理からある$n'$があって$a^{n'}\cap b \subset a^nb$となる。

$I$がinjectiveだから$b/(b \cap a^n) \to J \to I$は$A/a^{n'} \to I$に伸び、$J$の定義から$A/a^{n'} \to J$を定める。

このことから前の全射が言える。
\end{proof}

flasqueなら$\Gamma$-acyclicなので、cohomologyが計算できる。

\section{Projective spaceのcohomology}
$S$をgraded ringとし$M$をgraded $S$-modとする。
$M(i)$を$M$の次数シフト、つまり$M(i)_d=M_{i+d}$で定まるgraded $S$-modとする。

$\Proj X$上の$O_X$-mod $\tilde{M}$を$\tilde{M}\vert_{D_+(f)}=\tilde{(M_{(f)})}$で定まる層とする。
とくに$\tilde{S(n)}=O_X(n)$とかく。

$P^n$の$O(d)$のcohomology $H^i(P^n, O(d))$を計算する。
次元$n$と次数$d, i$について帰納的に。
完全列$0 \to F(-1) \to F \to F\vert_H \to 0$を使う。

Cech cohomologyで計算する。

$X=P^1$について直接計算してみる。
$X=U_0 \cup U_1, U_0=D_+(x_0), U_1=D_+(x_1)$とaffine openで被覆する。
これに対してCech複体は$0 \to \Gamma(U_0, \cF) \times \Gamma(U_1, \cF) \to \Gamma(U_0\cap U_1, \cF)\to 0$である。
これのcohomologyを計算する。

\section{Serre duality}
adeleでの記述、相互法則、留数定理

residueによりdualityが与えられること。
$(\omega,\theta)$がdualizing pairとは、$\omega$がcoherent $O_V$-modで$\theta:H^1(V,\omega) \to k$であり、
任意のcoherent $O_V$-mod $F$に対して

\begin{align*}
Hom(F,\omega)\to Hom(H^1(V,F),H^1(V,\omega))\to Hom(H^1(V,F),k)
\end{align*}
が同型になること。

特に$F=L$が可逆層のとき、$H^0(L^{-1}\otimes\Omega)\cong H^1(L)^\vee$を定める。
$(\Omega_{V/k}, \int_V)$がdualizing pairであることを証明する。

\section{Riemann-Roch}

\section{residue}
Tateの論文
JOHN TATE, Residues of differentials on curves
\begin{dfn}
$V$を$k$ベクトル空間とし、$A, B$をその部分空間とする。
$A<B$とは$A+B/B$が有限次元であること、$A\sim B$とは$A<B$かつ$B<A$であること。

$A\subset V$に対し、$E, E_0, E_1, E_2\subset End(V)$を
\begin{enumerate}
\item $\theta\in E$は$\theta A<A$
\item $\theta\in E_1$は$\theta V<A$
\item $\theta\in E_2$は$\theta A<0$
\item $E_0=E_1\cap E_2$
\end{enumerate}
として定義する。
\end{dfn}

\begin{dfn}[trace]
$k$線形写像$\theta:V \to V$がfinite potentとはある$n$が存在して$\theta^nV$が有限次元であること。
このとき、$Tr_V\theta$を以下のように定義する。
\begin{enumerate}
\item $V$が有限次元の場合には、通常のtraceとする。
\item $W\subset V, \theta W\subset W$の時には$Tr_V\theta=Tr_W\theta+Tr_{V/W}\theta$とする。
\item $\theta$がnilpotentなら$Tr\theta=0$とする。
\end{enumerate}
$\theta$がfinite potentのとき、$W=\theta^nV$とすると$Tr_W\theta$は定義可能で、$Tr_{V/W}=0$となるから、これで定義できる。
$n$に依存せず決まるか?
\end{dfn}

\begin{thm}[definition of residue]
$res^V_A:\Omega^1_{K/k}\to k$で以下を満たすものが唯一存在する。
$f,g\in K$に対し、$f=f_1\mod E_2, g_1=g\mod E_2$であり、$f_1\in E_1$または$g_1\in E_1$なるものに対し、
$res^V_A(fdg)=Tr_V([f_1,g_1])$が成り立つ
\end{thm}

\section{Poisson和公式}
Serreの本、TateのSerre duality for non-closed $k$、Ramacrishnaの本

$k=\F_q$とし、$X/k$をsmooth projective curveとし$\A$を$X$のアデールとする。
$\psi:k(X)\backslash \A \to \C^\times$を固定することで、位相群としての双対
\begin{align*}
\A \to \hat{\A}; a\mapsto(x\mapsto\psi(ax)
\end{align*}
が定まる。

この$\psi$の構成に$X$上の$1$-form $\omega$を用いる。
$\mu$を$\A$のHaar measureとする。

\begin{prop}
$f\in L^1(\A)$かつ$\hat{f}\in L^1(\hat{\A})$とする。
このとき、ある$c\in\R^\times$があって$\hat{\hat{f}}=cf$となる。
\end{prop}
$f=1_U, U=\prod_xO_x$とすると、$c=\mu(c)\mu(D), D=\psi^{-1}(1)$となる。この$D$をconductorという。

\begin{thm}[Poisson和公式]
$f\in L^1(\A)$であって$\sum_{\xi\in k(X)}\abs{f(x+\xi)}$が広義一様収束するとする。
$\sum_{\xi\in k(X)}\abs{\hat{f}(\xi)}$が収束するなら
\begin{align*}
\sum_{\xi\in k(X)}f(\xi)=\sum_{\xi\in k(X)}\hat{f}(\xi)
\end{align*}
\end{thm}

$\psi$に対して$n_x\in\Z$を$\psi_x(p_x^{n_x})=1$かつ$\psi_x(p_x^{n_x-1})\neq1$なるものとし、
\begin{align*}
(\psi)=\sum_{x\in X}-n_x[x]
\end{align*}
と定める。これは$\psi$の取り方によらず定まる。

$f=\otimes 1_{O_x}$とする。
$X$のdivisor $D$に対して$x(D)\in\A$を$v_x(x(D)_x)=n_x(D)$となるものとして定め、
$h(y)=f(yx)$に対してPoisson和公式を使うと
\begin{align*}
\sum h(y)=\sum\hat{h}(y)
\end{align*}
となる。
\begin{align*}
\sum h(y)=\sum f(xy)=\sum_{y\in k(X), v_x(xy)\geq0} 1=q^{l(D)}
\end{align*}
となり、
\begin{align*}
\sum\hat{h}(y)=\sum\int_{\A}f(zx)\psi(yz)dx\\
=\sum\frac{1}{\abs{x}}\int_{\A}f(z)\psi(yzx^{-1})dz\\
=\sum_y\frac{1}{\abs{x}}\hat{f}(yx^{-1})\\
=q^{\deg(x)v_x(\psi)/2}
\end{align*}
となる。
このことからRRが証明できる。
ここでcountingの代わりにmotivicでやれば一般の$k$で同様にRRができる?LoeserとかHrushovskiとか?

$\A$の双対性を$\omega$を用いて証明する。(代数隊の場合は?)
$X=\P^1$の場合について構成すれば、一般には$Y/X$のtraceを用いて証明できる。

\begin{align*}
\Omega_{\F_q(t)/\F_q}\to(\A_{\F_q(t)}/\F_q(t))^{\wedge}\\
\omega\mapsto((r_x)\mapsto\sum_x\res(r_x\omega))
\end{align*}
ここで$\res_x(\sum_ia_i\pi^id\pi)=Tr_{k(x)/k}(a_{-1})$により定める。
\end{document}